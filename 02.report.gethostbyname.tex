\documentclass[12pt]{article}

\usepackage{geometry}
\usepackage{parskip}
\usepackage{tikz}

% This style is used to create block diagrams, you'll find it useful since many of your figures would be of that form, I'll try add more styles in the future :)
\usetikzlibrary{trees,positioning,fit,calc}
\tikzset{block/.style = {draw, fill=blue!20, rectangle,
                         minimum height=3em, minimum width=4em},
        input/.style = {coordinate},
        output/.style = {coordinate}
}

\usepackage[section]{minted}
\usepackage{xcolor}
\usemintedstyle{porland}

\usepackage{chngcntr}
\counterwithin{figure}{section}

\usepackage{tocbasic}
\setuptoc{lol}{levelup}

% \usepackage{indentfirst}
\geometry{a4paper, margin=1in}

%----------EDIT COVER INFO HERE -----------------%

\def \LOGOPATH {logo/logo-usth.png}
\def \DEPARTEMENT {Department of Information and Communication Technology}
\def \COURSENUM {ENCS101}
\def \COURSENAME {Network Programming}
\def \REPORTTITLE {C Program to resolve domain names}
\def \STUDENTNAME {NGUYEN Viet Hung}
\def \STUDENTID {BI11-103}
\def \INSTRUCTOR {TRAN Giang Son}

%------------------------------------------------%

\setlength{\parindent}{2em}
\setlength{\parskip}{0em}

\begin{document}

\pagenumbering{Roman}

\begin{titlepage}
    \vfill
    \begin{center}
        \includegraphics[width=0.7\textwidth]{\LOGOPATH} \\
        \hfill \\
        \Large{\DEPARTEMENT} \\
        \Large{\COURSENAME} \\
        \vfill
        \textbf{\LARGE{\REPORTTITLE}}
    \end{center}
    \vfill
    \begin{flushleft}
        \Large{\textbf{Prepared by:} \STUDENTNAME\;-\;\STUDENTID} \\
        \Large{\textbf{Instructor:} \INSTRUCTOR} \\
        \Large{\textbf{Date:} \today}
    \end{flushleft}
    \vfill
\end{titlepage}

%--------------ABSTRACT ------------------------%
{
\section*{\centering Abstract} 
This document specifies a method and procedures to implement a DNS resolver program in C language. Specifically, two functions C functions gethostbyname() and inet\_ntop() are used to resolve domain name to IP addresses and present the result.
\clearpage
}

%-----------------------------------------------%

\tableofcontents
\clearpage

\setlength{\parskip}{\baselineskip}%

\pagenumbering{arabic}

%--------------INTRODUCTION ---------------------%

\section{Deploy practical work to VPS:}

\subsection{Access to server throught sptp using ssh-key:}

\begin{itemize}
    \item Domain name comes from user input:
    \begin{verbatim}
viethung@viethung-thinkpad:~$ sftp -i .ssh/keyggcloud 
hungnv_bi11_103_st_usth_edu_vn@35.222.105.161
    \end{verbatim}
    \item Output:
    \begin{verbatim}
viethung@viethung-thinkpad:~$ sftp -i .ssh/keyggcloud 
hungnv_bi11_103_st_usth_edu_vn@35.222.105.161
Connected to 35.222.105.161.
sftp>
    \end{verbatim}
    \item Put practical work to server:
    \begin{verbatim}
viethung@viethung-thinkpad:~$ sftp -i .ssh/keyggcloud 
hungnv_bi11_103_st_usth_edu_vn@35.222.105.161
Connected to 35.222.105.161.
sftp> lcd ~/Desktop/B2NetwokPro/netprog2022
sftp> put prac2.c
Uploading prac2.c to /home/hungnv_bi11_103_st_usth_edu_vn/prac2.c
prac2.c                                       100%  601     2.7KB/s   00:00    
sftp> 
    \end{verbatim}
    \item Output:
    \begin{verbatim}
hungnv_bi11_103_st_usth_edu_vn@instance-1:~$ ls
prac2.c
    \end{verbatim}
\end{itemize}

\section{Run practical work on server and own PC:}

\subsection{Run practical work on server:}
\begin{verbatim}
hungnv_bi11_103_st_usth_edu_vn@instance-1:~$ ls
a.out  prac2.c
hungnv_bi11_103_st_usth_edu_vn@instance-1:~$ ./a.out usth.edu.vn
usth.edu.vn IPv4: 119.17.215.232
\end{verbatim}

\subsection{Run practical work on own computer:}
\begin{verbatim}
viethung@viethung-thinkpad:~/Desktop/B2NetwokPro/netprog2022$ ./a.out usth.edu.vn
usth.edu.vn IPv4: 119.17.215.232
\end{verbatim}

\section{Comapre result:}
The output result is the same on both cases:\Verb"119.17.215.232" for domain \Verb"usth.edu.vn"

\section{Explain:}
\subsection{Structure:}
\begin{verbatim}
Structure **struct in_addr**: Converts from struct form to string 
of dots-and-numbers form and vice-versa such as IP addr in cmd
\end{verbatim}

\subsection{Function gethostbyname():}
\begin{verbatim}
Function **gethostbyname()**: This **DNS** function belongs to **hostent** 
structure and that retrieves host information correspondingto a host name 
from database such as DNS. If the host name is invalid or not found, exit 
with error message:

if (host_info == NULL) {
        perror("gethostbyname");
        exit(1);
    }

To use gethostbyname(), we must include these library:

#include  <netdb.h>
#include <sys/socket.h>
\end{verbatim}

\subsection{Function gethostbyname():}
\begin{verbatim}
Function **inet_ntoa()**: This function belongs to **in_addr** structure.
**"inet_toa()"** converts a network address in **in_addr** to 
dots-and-numbers string such as: IPv4: 10.20.10.255. It returns non-zero 
if the address or domain is valid and returns zero if invalid one.

To use **inet_ntoa()**, we must include these library:

#include <sys/socket.h>
#include <netinet/in.h>
#include <arpa/inet.h>
\end{verbatim}

\subsection{Trace all IP for a domain:}
\begin{verbatim}
To list all the IP addr of a domain, we have to use loop to examine all the address value in address list:

for (int i = 0; host_info->h_addr_list[i] != NULL; i++) {
        address = (struct in_addr *)(host_info->h_addr_list[i]);
        printf("%s IPv4: %s\n", input, inet_ntoa(*address));
    }
\end{verbatim}

\clearpage


\end{document}
